\documentclass[12pt,a4paper,twoside,openright]{book}

\usepackage[italian]{babel}
\usepackage[utf8]{inputenc}

\usepackage{amsmath,amsfonts,amssymb,amsthm}
\usepackage{caption}
\usepackage[usenames]{color}
\usepackage{enumerate}
\usepackage{fancyhdr}
\usepackage{fancyvrb}
\usepackage{float}
\usepackage{graphicx}
\usepackage{hyperref}
\usepackage{indentfirst}
\usepackage{listings}
\usepackage{marvosym}
\usepackage{multicol}
\usepackage{sectsty}
\usepackage{subcaption}
\usepackage{tocloft}
\usepackage[table]{xcolor}
\usepackage{url}

%vedere come influirebbe su citazioni ecc.
%\hypersetup{
%    colorlinks,
%    citecolor=black,
%    filecolor=black,
%    linkcolor=black, %choose some color if you want links to stand out
%    urlcolor=black
%}
\hypersetup{
   hidelinks,
   linktoc=all     %set to all if you want both sections and subsections linked
}

\input{style/settings.tex}

\title{\textbf{NOME PROGETTO\\Caption progetto tipo una brevissima descrizione insomma}}
\author{Borficchia Davide, \hspace{5pt}Elia Di Pasquale, \\Pierfederici Eugenio, \hspace{5pt}Siboni Enrico}
\date{A.A. 2017/2018}

\makeindex

\begin{document}
\maketitle

\frontmatter 

\tableofcontents

%\input{introduction.tex}
	
\mainmatter

\pagestyle{fancy} 
\fancyhead[LE,RO]{\thepage}
\fancyfoot{}

\input{chapter1_development_process.tex}
\input{chapter2_requirements.tex}
\input{chapter3_architectural_design.tex}
\input{chapter4_detail_design.tex}
\input{chapter5_implementation.tex}
\input{chapter6_retrospective.tex}

%\input{conclusions.tex}
	
%\backmatter	

%\begin{thebibliography}{99}
\addcontentsline{toc}{chapter}{Bibliografia}

\bibitem{bib001} Autore1, Autore2,
{\em Titolo dell'opera},
Editore, Anno.

\bibitem{bib002} Autore1, Autore2,
{\em Titolo dell'opera},
Editore, Anno.

\bibitem{bib003} Autore1, Autore2,
{\em Titolo dell'opera},
Editore, Anno.

\end{thebibliography}

\end{document}