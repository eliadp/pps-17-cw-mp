\documentclass[12pt,a4paper,twoside,openright]{book}

\usepackage[italian]{babel}
\usepackage[utf8]{inputenc}

\usepackage{amsmath,amsfonts,amssymb,amsthm}
\usepackage{caption}
\usepackage[usenames]{color}
\usepackage{enumerate}
\usepackage{fancyhdr}
\usepackage{fancyvrb}
\usepackage{float}
\usepackage{graphicx}
\usepackage{hyperref}
\usepackage{indentfirst}
\usepackage{listings}
\usepackage{marvosym}
\usepackage{multicol}
\usepackage{sectsty}
\usepackage{subcaption}
\usepackage{tocloft}
\usepackage[table]{xcolor}
\usepackage{url}

%vedere come influirebbe su citazioni ecc.
%\hypersetup{
%    colorlinks,
%    citecolor=black,
%    filecolor=black,
%    linkcolor=black, %choose some color if you want links to stand out
%    urlcolor=black
%}
\hypersetup{
   hidelinks,
   linktoc=all     %set to all if you want both sections and subsections linked
}

\AtBeginDocument{%
	\renewcommand{\contentsname}{Indice}
	\renewcommand\tablename{Tabella}
	\renewcommand\figurename{Figura}
	\renewcommand{\lstlistingname}{Listato}
	\renewcommand{\refname}{Riferimenti}
}

\definecolor{dkgreen}{rgb}{0,0.6,0}
\definecolor{gray}{rgb}{0.5,0.5,0.5}
\definecolor{mauve}{rgb}{0.58,0,0.82}

\lstset{
  frame=single,
  captionpos=b,
  language=Java,
  aboveskip=3mm,
  belowskip=3mm,
  showstringspaces=false,
  columns=flexible,
  basicstyle={\small\ttfamily},
  numbers=none,
  numberstyle=\tiny\color{gray},
  keywordstyle=\color{blue},
  commentstyle=\color{dkgreen},
  stringstyle=\color{mauve},
  breaklines=true,
  breakatwhitespace=true,
  tabsize=3
}

\makeatletter
\def\cleardoublepage{
	\clearpage\if@twoside \ifodd\c@page\else
	\hbox{}
	\thispagestyle{empty}
	\newpage
	\if@twocolumn\hbox{}\newpage\fi\fi\fi
}

\makeatother

\setlength{\textwidth}{14cm}
\setlength{\textheight}{21cm}
\setlength{\footskip}{3cm}

\setlength{\hoffset}{0pt}
\setlength{\voffset}{0pt}

\setlength{\oddsidemargin}{1cm}
\setlength{\evensidemargin}{1cm}

\title{\textbf{NOME PROGETTO\\Caption progetto tipo una brevissima descrizione insomma}}
\author{Borficchia Davide, \hspace{5pt}Elia Di Pasquale, \\Pierfederici Eugenio, \hspace{5pt}Siboni Enrico}
\date{A.A. 2017/2018}

\makeindex

\begin{document}
\maketitle

\frontmatter 

\tableofcontents

%\chapter{Introduzione}
\markboth{INTRODUZIONE}{INTRODUZIONE}

Qui il testo dell'introduzione alla tesi. Generalmente l'introduzione non dovrebbe superare le 2/3 pagine e dovrebbe essere scritta solo alla fine.
	
\mainmatter

\pagestyle{fancy} 
\fancyhead[LE,RO]{\thepage}
\fancyfoot{}

\chapter{Processo di sviluppo}
\chapter{Requisiti}
\chapter{Design architetturale}
\chapter{Design di dettaglio}
\chapter{Implementazione}
\chapter{Retrospettiva}

%\chapter*{Conclusioni}
\addcontentsline{toc}{chapter}{Conclusioni}
\markboth{CONCLUSIONI}{CONCLUSIONI}

Qui il testo delle conclusioni alla tesi. Non deve essere un riepilogo di quanto fatto nella tesi ma piuttosto le conclusioni raggiunte relative al lavoro svolto.

	
%\backmatter	

%\begin{thebibliography}{99}
\addcontentsline{toc}{chapter}{Bibliografia}

\bibitem{bib001} Autore1, Autore2,
{\em Titolo dell'opera},
Editore, Anno.

\bibitem{bib002} Autore1, Autore2,
{\em Titolo dell'opera},
Editore, Anno.

\bibitem{bib003} Autore1, Autore2,
{\em Titolo dell'opera},
Editore, Anno.

\end{thebibliography}

\end{document}